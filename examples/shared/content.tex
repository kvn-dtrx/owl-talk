\section{Grundlegendes}

\begin{frame}{Listen-Umgebungen}
    Eine \texttt{itemize}-Umgebung:
    \begin{itemize}
        \item Erster Punkt.
        \item Zweiter Punkt.
              \begin{itemize}
                  \item Erster Unterpunkt des zweiten Punktes.
              \end{itemize}
    \end{itemize}

    Eine \texttt{enumerate}-Umgebung:

    \begin{enumerate}
        \item Erster Punkt
              \begin{enumerate}
                  \item Erster Unterpunkt des ersten Punktes.
              \end{enumerate}
        \item Zweiter Punkt.
    \end{enumerate}
\end{frame}

\begin{frame}{Zentrierter Text}
    Eine zentrierte, vielleicht auch zentrale Aussage:
    \begin{center}
        Ein Subjekt genau dann, wenn ein Objekt.
    \end{center}
    \pause\par%
    Nun die Fortsetzung nach der Digestation dieser tiefliegenden Aussage.
\end{frame}

\begin{frame}{Zweispaltiges Layout}
    Zweispaltig – aber nicht zwiespältig:
    \par{}~%
    \begin{columns}
        % How is exact alignment achievable at this point?
        \column[t]{0.485\textwidth} Das Subjekt \dots
        \column[t]{0.485\textwidth} \dots und das Objekt.
    \end{columns}
\end{frame}

\begin{frame}{Sukzessives Aufdecken}
    Die vier Wurzeln des Satzes vom Grunde lauten:
    \begin{itemize}[<+->]
        \item Werden
        \item Erkennen
        \item Anschauen
        \item Wollen
    \end{itemize}
\end{frame}

\begin{frame}{Zitat -- Kant im englischen Original}
    \begin{verse}
        As any dedicated reader can clearly see, the Ideal of practical reason is a representation of, as far as I know, the things in themselves; as I have shown elsewhere, the phenomena should only be used as a canon for our understanding.

        \hspace*\fill{\hspace*\fill{\small\normalfont---~Immanuel~Kant~(*1724, †1804), deutscher Philosoph}}
    \end{verse}
\end{frame}
\begingroup
\usebackgroundtemplate{%
    \includegraphics[
        width={\paperwidth},
        height={\paperheight},
    ]{example-image-empty}
}
\begin{frame}{Veränderter Hintergrund}
    Eine Folie mit verändertem Hintergrund.
\end{frame}
\endgroup

\section{Mathematisches}

\begin{frame}{Symbolreichtum}

    Lateinisch kann jeder \ldots

    \newfontface\greekfont{Gentium}
    Griechisch: {\greekfont α β γ δ ε ζ η θ ι κ λ μ ν ξ ο π ρ σ τ υ φ χ ψ ω}

    \newfontface\hebrewfont{Libertinus Serif}
    Hebräisch: {\hebrewfont א ב ג ד ה ו ז ח ט י כ ל מ נ ס ע פ צ ק ר ש ת}

    \newfontface\cyrillicfont{FreeSerif}
    Kyrillisch: {\cyrillicfont а б в г д е ж з и й к л м н о п р с т у ф х ц ч ш щ ъ ы ь э ю я}

    Mathematisch: $\displaystyle{ \int, \sum, \prod, \coprod, \biguplus }$
\end{frame}

\begin{frame}{Gleichungen}
    Eine einfache Gleichung:

    \begin{equation*}
        a^2 + b^2 = c^2
    \end{equation*}

    Eine einfache Gleichung mit Tag:

    \begin{equation}\tag{\ensuremath{\star}}
        \mathbf 1_{\{i\}}(j) = \delta_{ij}
    \end{equation}
\end{frame}

\begin{frame}{Mehrzeilige Rechnung}
    Eine Rechnung über mehrere Zeilen:
    \begin{align*}
        \onslide<2->{
            \alpha x^2 + \beta x
         & = \alpha \Biggl( x^2 + \frac{\beta}{\alpha} x \Biggr)
        \\ }
        \onslide<3->{
         & = \alpha \Biggl( x^2 + 2 \frac{\beta}{2\alpha} x + \left( \frac{\beta}{2\alpha} \right)^2 \Biggr) - \alpha \left( \frac{\beta}{2\alpha} \right)^2
        \\ }
        \onslide<4->{
         & = \alpha \Biggl( x + \frac{\beta}{2\alpha} \Biggr)^2 - \frac{\beta^2}{4\alpha}
        }
    \end{align*}
\end{frame}

\begin{frame}{Definition mit Warnung}
    \begin{mathblock}{Grenzwert einer Folge}
        \begin{description}
            \item [Prem.] $x = (x_n)_{n} \in \mathbb R^\mathbb N$.

            \item [Obs.] Es gibt höchstens ein $x_\infty \in \mathbb R$ mit
                  \begin{equation*}
                      \forall_\epsilon(\epsilon>0) \exists_{N = N(\epsilon)} (N \in \mathbb N) \forall_n (n \in \mathbb N, n \geq N): \| x_n - x_\infty \| < \epsilon.
                  \end{equation*}
            \item [Def.] Im Existenzfalle heißt $x_\infty$ der Grenzwert der Folge $x$.
        \end{description}
    \end{mathblock}
    \pause%
    \begin{alertblock}{Eine Warnung}
        Die entsprechende Aussage gilt nicht notwendig für nicht-Hausdorffsche topologische Räume!
    \end{alertblock}
\end{frame}

\begin{frame}{Ein Satz mit (schrittweisem) Beweis}
    \begin{mathblock}{Ein wichtiger Satz}
        Es gilt: Ein Subjekt genau dann, wenn ein Objekt.
    \end{mathblock}
    \pause%
    \begin{proof} \par
        {\bfseries Schritt 1:} Trivial.
        \par\pause%
        {\bfseries Schritt 2:} Ausgelassen, zu technisch für eine Folie.
        \par\pause%
        {\bfseries Schritt 3:} Unmittelbar aus dem ersten und zweiten Schritt.
    \end{proof}
\end{frame}

\section{Sonstiges}

\begin{frame}{Speaker Notes}{}
    Die Magie dieser Folie \ldots
    \par{}~\pause%
    \ldots kennt zunächst einmal nur der Quelltext.
    \note{So wie beispielsweise diese Anmerkung.}
\end{frame}

\section[\ldots Titel in der Kopfleiste]{Titel im Inhaltsverzeichnis \ldots}

\begin{frame}{Titeländerung}{}
    Die Abschnittstitel können für das Inhaltsverzeichnis und die Kopfleiste separat gesetzt werden.
\end{frame}

% Endmarkierende Folie

\begin{standoutframe}
    \begin{center}
        {\color{accentcolor}\fontsize{60}{60}\selectfont ω}
    \end{center}
\end{standoutframe}

\appendix

\begin{frame}{Das Ende?}
    Geht es nach dem Ende vielleicht weiter? Aber dann wäre es doch nicht das Ende!
\end{frame}

\begin{frame}{Literatur}
    \begin{thebibliography}{16}
        \bibitemonline{Agerius2025}[Agerius2025] \textsc{Aulus Agerius}: \url{https://example.org/infinite-scrolling} (zuletzt gestern abgerufen)

        \bibitembook{Maevius1732}[Maevius1732] \textsc{Publius Maevius}: \textit{De Arte Nihili}, Imaginaria Press (1732)

        \bibitemarticle{Titius1901}[Titius1901] \textsc{Lucius Titius}: \textit{On the Metaphysical Weight of Shadows at Noon}, Journal of Speculative Studies 42 (1901)
    \end{thebibliography}
\end{frame}
